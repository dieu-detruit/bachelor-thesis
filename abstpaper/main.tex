\documentclass[a4j]{jarticle}
\addtolength{\topmargin}{-2cm}
\addtolength{\textheight}{4cm}
\addtolength{\textwidth}{2cm}
\addtolength{\oddsidemargin}{-1cm}
\addtolength{\evensidemargin}{-1cm}

\begin{document}

\twocolumn[
\begin{center}
	{\Large 位相回復法を用いた天文用Wolterミラーの非接触形状計測装置の開発}
\end{center}
\begin{center}
	{\Large Development of Non-contact Measurement Method \\ for X-ray Telescope Mirror Using Ptychographical Method}
\end{center}

\begin{flushright}
	{\large 03-190395 渡辺 貴史} \\ % author
	{\large 指導教員 三村 秀和 准教授} % teacher
\end{flushright}

\begin{center}
 {\bfseries 概要}
\end{center}

天文分野において太陽の活動を観測することは非常に重要である。
これまで、活動領域のコロナによって形成された高温プラズマが発するX線領域の電磁波を計測することで、太陽の活動を明らかにする試みが行われてきた。
観測に利用されるWolter I型のミラーについて、その加工精度を高めることで撮影される太陽コロナの像がより鮮明になることが期待される。
本研究では、高精度に加工されたミラー内面を傷つけない形状測定装置を位相回復法を応用して新たに提案し、シミュレーションおよび実際に加工されたミラーの計測実験によって手法検討を行った。
\vspace*{2em}
]
%\thispagestyle{empty}

\section{序論}
天文分野において太陽の活動を観測することは非常に重要である。
これまで、活動領域のコロナによって形成された高温プラズマが発するX線領域の電磁波を計測することで、太陽の活動を明らかにする試みが行われてきた。

\section{}
卒論自体は章構成全体で起承転結を成すようにする。流れの一例として
は,

\section{ミラー計測実験}

\bibliography{reference}
\bibliographystyle{junsrt}

\end{document}
