\chapter{結論}
\thispagestyle{empty}
\label{chap6}
\graphicspath{{chap6/figure/}}
\minitoc

\newpage
%%%%%%%%%%%%%%%%%%%%%%%%%%%%%%%%%%%%%%%%%%%%%%%%%%%%%%%%%%%%%%%%%%%%%%%%%%%%%


% ================================================== %
% section
% ================================================== %
\section{提案手法の適用可能性}
\label{chap6_conclusion}

提案された手法による計測実験では、入射ビームの非点収差が極端に大きく、ミラー形状に起因する波面誤差を正しく測定することが出来なかった。
一方で、入射ビームの非点収差成分が位相回復計算によって算出され、そこから計算される焦点面強度分布が実際の計測強度分布と非常に良く一致したことから、計測手法および位相回復計算には大きな誤謬が無いことが示された。
入射ビームを再調整し、再度計測実験を行うことで、ミラーの形状誤差を反映した位相分布が計測されることが期待される。
また、ミラーを光軸周りに回転して複数回の計測を行い、平均化を行うことで、入射光由来の誤差成分を計算することができる。
違う方法として、加工誤差がほとんど無く理想的な集光性能を持つレンズによって同様の計測を行い、入射ビームに起因する収差を算出することも可能であると考えられる。
いずれにせよ、入射ビーム由来の収差を算出し、上流光学系を校正した上で再度実験することが望ましい。

\section{今後の展望}
\label{chap6_futureworks}

不等間隔の穴を配置したオブジェクトを下流端面で走査するという本研究の提案手法は、Nested Wolterミラーにも適用可能であると考えられる。
今後、シミュレーションによる検証をした上で、レンズ等を用いた擬似的な計測実験によって手法適用が可能かどうかを判定する。
Nested Wolterミラーを構成後、振動実験での測定などで本手法が適用できれば、開発の効率は格段に上昇すると言える。


\clearpage
\newpage


%%%%%%%%%%%%%%%%%%%%%%%%%%%%%%%%%%%%%%%%%%%%%%%%%%%%%%%%%%%%%%%%%%%%%%%%%%%%%
%%% Local Variables:
%%% mode: katex
%%% TeX-master: "../thesis"
%%% End:
